\begin{abstract}
Across almost all scientific disciplines, the instruments that record our experimental data and the methods required for storage and data analysis are rapidly increasing in complexity.
This gives rise to the need for scientific communities to adapt on shorter time scales than traditional university curricula allow for, and therefore requires new modes of knowledge transfer.
The universal applicability of data science tools to a broad range of problems has generated new opportunities to foster exchange of ideas and computational workflows across disciplines.
In recent years, hack weeks have emerged as an effective tool for fostering these exchanges by providing training in modern data analysis workflows.
While there are variations in hack week implementation, all events consist of a common core of three components: tutorials in state-of-the-art methodology, peer-learning and project work in a collaborative environment.
In this paper, we present the concept of a hack week in the larger context of scientific meetings and point out similarities and differences to traditional conferences.
We motivate the need for such an event and present in detail its strengths and challenges.
We find that hack weeks are successful at cultivating collaboration and the exchange of knowledge.
Participants self-report that these events help them both in their day-to-day research as well as their careers.
Based on our results, we conclude that hack weeks present an effective, easy-to-implement, fairly low-cost tool to positively impact data analysis literacy in academic disciplines, foster collaboration and cultivate best practices.
\end{abstract}
