\section*{Design considerations}

Design considerations depend very strongly on goals.
In general, longer events allow for a larger taught component, more ambitious projects and especially for cross-disciplinary events are more likely to provide enough time for different groups to effectively communicate across barriers of professional terminology.
On the other hand, because of their large participatory component, hack weeks tend to be exhausting, and long events may lead to fatigue among attendees, where active participation and measurable outcomes may drop sharply in the latter days of the workshops.
%This is particularly important if---as is often the case---participants continue participating in workshop-relevant activities even after the official component ends for the day.
Ways to mitigate these effects include capping taught components during the day, providing a reasonable clear schedule, and limiting parallel components to avoid decision fatigue.
%At Astro Hack Week, we have limited the scheduled taught components to no more than three out of nine hours for each of the five workshop days, and unscheduled (shorter) tutorials to a maximum of two per day.
%Even so, in our surveys, participants occasionally report concerns about choosing to attend tutorials versus working on projects.

Aside from time, space plays a particularly important role in facilitating a successful hack week.
Universities in general provide a convenient venue with existing structures to facilitate hack weeks (e.g. access to scientific publications, institutional support including staff and funding).
On the other hand, most spaces in universities are designed for lectures, which are diametrically opposed to unconference-style events.
Thus, finding an appropriate physical space within a university may be a challenge.
However, with the rise of active learning as a preferred teaching methods, traditional lecture spaces are transforming into more flexible spaces that are generally appropriate for hack weeks.
As a general rule, the smaller a hack week's emphasis on a taught component, the more flexible the space has to be, with ample opportunity for re-configuration.
%At Astro Hack Week, we have found it beneficial to spend one day mid-week at a different location (e.g. a company headquarter) to engage participants and break the routine.
%It is worth noting that collaborative spaces for non-traditional workshops are becoming more prevalent in academia (some attached to universities, some not) and may provide support and infrastructure that a traditional university location may not.

Another important design consideration is group size.
%With all previous hack weeks severely oversubscribed, it seems natural to simply admit more participants.
%However, this can counter the ideals of the workshop, in particular when building a community is one of its stated goals.
If the group is too large, participants are unlikely to even meet each other, and workshop cohesion may be lost as the workshop fractures into smaller groups, often among participants who already know each other.
This may inhibit knowledge transfer by clustering participants into small in-groups.
%Additionally, it is likely that the number participant-led components in the schedule may increase with workshop size.
%While generally desirable, a programme that is too crowded may lead to fatigue
On the other hand, if the size of the workshop is too small, it is unlikely to achieve the desired diversity among participants to foster new collaborations across sub-fields and disciplines.
In the past, we have found groups with sizes between 50 and 70 participants to be large enough to encourage a breadth of projects while allowing the workshop to function as a cohesive group.

As mentioned above, the balance between pedagogy and working depends both on the goals of the workshop and the topics around which the workshop is organized.
If participants have little shared knowledge, more teaching may be necessary in order to allow participants to effectively communicate with each other.
In communities where a shared understanding exists, tutorials can focus on more advanced or innovative topics, and less time may be allocated for them, leaving more time for active participation.
%In astronomy, a relatively small field, even students generally share a common knowledge base and have rudimentary knowledge about the types of data used and the challenges associated with each.
%Thus, we have focused the taught component of Astro Hack Week less on domain-specific knowledge, and instead offered tutorials in topics from domain-adjacent fields like statistics and computer science that attendees are unlikely to have encountered in their regular education.

Hack week outcomes, in turn, depend strongly on participants and are often a function of their interests and seniority.
some attendees arrive with the stated goal of writing a specific scientific article, usually more advanced participants with significant pre-knowledge of both hackathons and their topic of interest.
Many attendees arrive with the plan to learn a specific topic (such as machine learning) or bring a specific data set they believe the new knowledge may be applicable to.
This leads to a wide variety of project types from sandbox-style explorations to focused work efforts.
%It is worth noting that while a scientific paper need not be the stated goal of a hack (and is unlikely to be completed in the short time allocated in any case), results may still be published as short reports or unconference proceedings.
%For example, Neurohack week provides a venue for participants to publish a short (two-page) ''project report`` summarizing the hack that participants did during the week of NHW.
%Similarly, Python in Astronomy gathers all documents produced during the workshop (unconference transcripts, talk summaries, descriptions of sprint and hack projects) into citeable unconference proceedings.

\subsection*{Box 1: Impostor Syndrome}

The \textit{impostor phenomenon}, or \textit{impostor syndrome} (IS) is a dissonant feeling experienced by certain high-achieving individuals, that despite objective evidence to the contrary, they are in fact not as intelligent or capable as they appear.
Individuals with IS thus experience a fear of being ''found out``, shamed and expelled from their environment \cite{Clance1978-ef}.
Initial observations of the first Astro Hack Week conducted by data science ethnographer Brittany Fiore-Gartland\footnote{\url{http://astrohack week.org/blog/ethnographic-notes.html}} suggested that hack weeks are an environment prone to a particular kind of IS: participants might feel the need to be experts in multiple aspects of the activities pursued during a hack week: expertise in a scientific domain, as well as expertise in a the variety of technical tools used.
This particular form of IS hinges to some degree on the design focus on diversity of backgrounds (everyone else seems to know something that you do not!) and might be further exacerbated by the expectation that attendants expose their ideas to public scrutiny, find collaborators in a very short amount of time, and not least produce and present a successful hack at the end of the week\footnote{see also this insightful blog post: \url{https://medium.com/astronomy-without-stars/the-horror-of-hack-days-52c6b52cfc3b}}.
The prevalence of IS at a hack week may be endemic to the format, and should thus be a major concern for any organizing committee.
This is because of the chilling effect it tends to have on participants and the community as a whole, and particularly on women (in particular in fields in which women are under-represented) and members of ethnic and racial minorities, correlating with anxiety and other forms of mental distress \cite{Parkman2016-ro}.
Less severely, another major concern is that IS inhibits risk taking: participants experiencing it will be less likely to ask a question, to put forward an idea for a hack, to be pro-active about forming new collaborations.
 Many of the goals of a hack week, including the successful completion of projects, lateral knowledge transfer, as well as community building are hampered.
We are working within our hack weeks to mitigate IS using various techniques.
With respect to minority participants, ensuring adequate representation can decrease feelings of otherness and may help reduce IS.
More generally, being open about the presence and prevalence of IS can help participants feel more at ease.%(e.g. one Astro Hack Week participant remarked in a survey, \textit{''I really appreciated the direct acknowledgement of impostor syndrome on the first day.
%I think it helped ease the feeling!!``}).
Additionally, role models are very effective at encouraging positive behaviour: asking participants with prior experience at hack weeks at all academic levels to ask questions, even when they might know the answer, can help foster an inclusive environment that rewards risk taking.
