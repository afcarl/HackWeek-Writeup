\section*{Why run a Hack Week?}

There are several reasons to run a hack week of the sort described here.

\begin{itemize}
\item{\textit{Education and training}: %Some hack weeks are more focused on education than others (see Figure 1).
Even those hackathons not explicitly focused on skills-development will include extensive discussions on reproducible and open science practices that are valuable to participants.
%Discussing these topics in a diverse group setting helps to ingrain within participants the basic conventions of the broader open source community.
Participants can then serve as ''ambassadors`` of these practices and foster their adoption by other researchers in the field.
Furthermore, the kind of co-working that is inherent to the hack week format facilitates lateral knowledge transfer, where participants learn from each other about practices and tools of the trade.}
%Often these tools are not described in research products such as papers, or even in the software implementations.}
\item{\textit{Tool development}: hack weeks present an opportunity for computational tool developers to get together with users of these tools, and to critically evaluate the use of the tools in addressing particular scientific questions.}
\item{\textit{Community building}: a hack week is an opportunity to galvanize a community within a field.% that cares about computational issues, and relates to specific open-source software projects.
For example, because of its focus on free and open-source tools, Neurohack week helps galvanize the Neuroimaging in Python (http://nipy.org) community.
It also allows researchers who may otherwise be isolated within their departments and subfield on account of their computational specialization to come together and exchange experiences.}
%As one participant at Astro Hack Week expressed in the survey, \textit{''The main thing I took from Astrohack week is the fact that I'm not alone``}.}
\item{\textit{Interdisciplinary research}: a short time-bounded event is an excellent opportunity to experiment with concepts, questions and methods that traverse boundaries within a discipline (e.g.\ between researchers using different methods, or studying different systems, or experimental models), as well as across disciplines.
Despite the impact of interdisciplinary research \cite{Hall2012-hi}, these experiments are often discouraged in traditional disciplinary research, because of the difficulty to achieve results in both disciplines, and because of the risks inherent in taking a less trodden path\footnote{\url{https://www.ncbi.nlm.nih.gov/labs/articles/12970550} and \url{https://www.researchgate.net/publication/8126355_EDUCATION_Risks_and_Rewards_of_an_Interdisciplinary_Research_Path}}.}
%This is despite the fact that interdisciplinary research can also be very impactful \cite{Hall2012-hi}.
%A diversity of research backgrounds, and interactions with scientists from other disciplines, further facilitates these experiments.}

\item{\textit{Recruitment and networking}: hack weeks often bring together research, government and industry communities, providing an opportunity for participants to learn about each other's interests and their abilities in close quarters.
This kind of extended opportunity for interaction and ''on-the-job`` testing is considered an excellent form of evaluation of prospective job candidates [CITATION NEEDED].
Conversely, participants have an opportunity to gain exposure to potential career opportunities in communities other than their own.}
%As an example, one of the participants in Neurohack week 2016 will be joining the eScience Institute as a post-doc in Fall of 2017 (and as an instructor in Neurohack week 2017).}
%\item{\textit{It's fun}: the opportunity to focus for a week on a slightly more whimsical and adventurous sort of scientific activities can be both fruitful, and enjoyable.
%Even aside from the prospect of interdisciplinary research, it can provide a laboratory for more experimental approaches and high-risk projects than scientist will take on in their day-to-day research.}
\end{itemize}

It is worth noting that the reasons for participants to attend a hack week are as diverse as the goals that the organizing committee explicitly states.
This is largely a function of the diversity of the participants (see section below), and should be something that organizers design for.
Beginner participants may attend primarily to learn a new technique or method, while more experienced participants attend to gain more experience mentoring and teaching.
Some researchers may come with particular projects and collaborators in mind, while others come with a focus on learning and with no explicit plan for the project-based work.
%While it is very hard to design a workshop that is universally useful to all participants, in practice many researchers attend with the understanding that some components will be more congruent with their own goals than others, but also bring an open mind and a willingness to learn.
%In addition, the diversity of goals can effectively be a strength of the workshop, if the organizers can facilitate matching up participants with complementary goals.
