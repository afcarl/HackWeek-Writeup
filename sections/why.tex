\section*{Why run a Hack Week?}

There are several reasons to run a hack week of the sort described here.

\begin{itemize}
\item{\textit{Education and Training}: %Some hack weeks are more focused on education than others (see Figure 1).
While some hack weeks are focused more on education than others (Figure 1), there is often a skill-development component that entails extensive discussion on reproducible research and open science practices. Participants gain a strong foundation in open science practices from the diverse group setting and go on the become ambassadors for such practices in their respective fields. This type of lateral knowledge transfer is a core attribute of a hack week, and provides an opportunity to learn skills that are not described in papers and software implementations.}

\item{\textit{Tool Development}: Hack weeks present an opportunity for scientific software developers to meaningfully engage with users and critically evaluate applications to particular scientific issues.}

\item{\textit{Community Building}: Hack weeks provide a tremendous opportunity to catalyze community development through a shared interest in solving computational challenges with open source software. These events allow computationally minded researchers to break from the isolation of their academic departments and build connections and spark new collaborations.}


\item{\textit{Interdisciplinary research}: Intensive, time-bounded collaborative events are an excellent opportunity to experiment with concepts, questions, and methods that span boundaries within and across disciplines. Despite the fact that such interdisciplinary experiments are highly impactful \cite{Hall2012-hi}, they are often discouraged in risk averse traditional academia \footnote{\url{https://www.ncbi.nlm.nih.gov/labs/articles/12970550} and \url{https://www.researchgate.net/publication/8126355_EDUCATION_Risks_and_Rewards_of_an_Interdisciplinary_Research_Path}}}.

\item{\textit{Recruitment and Networking}: Hack weeks are often a melting pot of participants from academia, government, and industry and provide numerous opportunities for networking. Close collaboration in diverse groups exposes skills that might be suitable for careers outside of one's narrow domain.}

\item{\textit{It's fun}: Hack weeks provide a respite from day-to-day research activities and provide a low-stress venue to learn new skills and attempt high-risk projects.}

\end{itemize}


It is worth noting that the reasons for participants to attend a hack week are as diverse as the goals that the organizing committee explicitly states.
This is largely a function of the diversity of the participants (see section below), and should be something that organizers design for.
Beginner participants may attend primarily to learn a new technique or method, while more experienced participants attend to gain more experience mentoring and teaching.
Some researchers may come with particular projects and collaborators in mind, while others come with a focus on learning and with no explicit plan for the project-based work.
%While it is very hard to design a workshop that is universally useful to all participants, in practice many researchers attend with the understanding that some components will be more congruent with their own goals than others, but also bring an open mind and a willingness to learn.
%In addition, the diversity of goals can effectively be a strength of the workshop, if the organizers can facilitate matching up participants with complementary goals.
