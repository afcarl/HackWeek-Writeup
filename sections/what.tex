\section*{What is a hackathon?}

Hackathons are time-bounded, collaborative events that bring together participants around a shared challenge or learning objective \cite{Decker2015}.
Hackathons originated from the open-source software movement, and have historically focused on software development, particularly in the form of coding sprints, and technology design as a way to motivate innovation, eventually being adopted also within the technology industry.
In recent years, hackathons have expanded into a model for intensive short-term collaboration across disciplinary and topical boundaries.
In addition, because of their focus on participatory engagement, hackathons provide numerous opportunities to `learn by doing' within a constructivist educational framework \cite{Bransford2000-lu,Papert1980-fh}.
With this in mind, hackathons around scientific topics, designed to foster collaboration \cite{Groen2015-cj,Moller2013-ah}, or provide an opportunity to learn \cite{Kienzler2015-zu,Lamers2014-xf}, are becoming more common.

However, in addition to these goals, core element of all hackathons include opportunities for networking, strengthening of social ties, and the building of community connections, both within and across disciplines.
Building on these core elements, there are various implementations of the hackathon concept with respect to the overall purpose, mode of participation, style of work environment and motivation \cite{Drouhard2017}:
``Catalytic'' hackathons seek novel project ideas aimed at solving a tractable, well-defined challenge.
``Contributive'' hackathons seek to improve to an existing effort through focused work on discrete tasks, for example to make up for deficiencies in an ongoing project.
Finally, ``Communal'' hackathons place a strong focus on building a culture of practice and developing resources within an existing community, often defined by a specific domain of knowledge.

Our past hack week events follow most closely the communal hackathon model, as it applies to scientific communities of practice.
Our approach aims to combine structured, tutorial-style instruction with informal education and peer learning opportunities occurring within projects and hacks.
Within the communal model we see these tools being implemented across a spectrum of approaches, the design of which depends on the specific characteristics of each community of practice (Figure \ref{fig:hackspectrum}).
For example, the astronomy community is relatively small and has a foundation of shared approaches and software implementations, allowing for a greater focus on project work over formal tutorials.
In contrast, both the neuro- and geoscience communities covered a broader range of sub-disciplines and had a less cohesive set of existing practices, calling for greater focus on tutorials and education.

The hack week model sets itself apart from other events notably in the goals that are pursued (see also Figure \ref{fig:hackspectrum}). Many industry hackathons are focused on delivering solutions to a specific problem or challenge on a very short timescale. Hack weeks are generally more open-ended than these hackathons and allow for a wider variety of outcomes and goals among participants. Within academia, the majority of venues for scientific collaboration, learning and communication fall into one of three categories: traditional conferences, summer schools, and, more recently, unconferences. All three types share elements with hack weeks, but have goals that are more narrowly defined. Traditional conferences are focused on the exchange of recent results, as well as networking. However, collaborative components are usually absent, or restricted to short time intervals (e.g.\ coffee breaks and ``hack days''). Unconferences are less structured and allow for greater input from participants, but are generally less focused on knowledge transfer and learning, but on achieving measurable outcomes. However, these outcomes are often in the form of policies, planning and white papers, whereas a hack week encourages participants to directly produce outcomes such as a paper, a visualization, or code. 

Three design considerations distinguish the hack week from the summer school model: (1) hack week transfer knowledge across many levels of seniority, with participants routinely leading tutorials, in part due to the relative newness of the content and the distributed knowledge base; (2) hack weeks contain a significant amount of unstructured time, creating the conditions for improvised exchanges and project development; and (3) hack weeks intentionally priortize participant diversity because this is known to maximize the likelihood for the co-creation of learning during the event.  

We note that the terminology for these events is constantly evolving, and that the ``hackathon'' concept may have implicit connotations that are disfavored in some communities.
One criticism of hackathons is that they propel the ``geek'' stereotype and may present a barrier to creating an inclusive working environment, especially for individuals traditionally underrepresented in science and technology \cite{Decker2015}.

This criticism needs to be actively addressed both through participant selection (see below), and by addressing the possibility that some participants may experience an ``imposter syndrome'' within such circumstances \cite{clance1978imposter} (see supplementary material).
%Also, while many industry hackathons are competitive, with teams actively competing to solve the same problems for prizes, academic hackathons, and specifically the hack weeks that we have organized are not explicitly competitive.
