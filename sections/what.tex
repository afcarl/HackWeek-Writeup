\section*{What is a hackathon?}

Hackathons are time-bounded, collaborative events that bring together participants around a shared challenge or learning objective \citep{Decker2015}.
Hackathons have historically focused on software development and technology design as a way to motivate innovation within industry.
In recent years, hackathons have expanded into a model for intensive short-term collaboration across disciplinary and topical boundaries.
In addition, because of their focus on participatory engagement, hackathons provide numerous opportunities to 'learn by doing' within a constructivist educational framework \citep{Bransford2000-lu,Papert1980-fh}.
With this in mind, hackathons around scientific topics, designed to foster collaboration \citep{Groen2015-cj,Moller2013-ah}, or provide an opportunity to learn \citep{Kienzler2015-zu,Lamers2014-xf}, are becoming more common.

%The recent surge in popularity of these events has resulted in a broad spectrum of ways to define the hackathon.
Following the typology of \citep{Drouhard2017}, we assert that core elements of all hackathons include opportunities for networking, the strengthening of social ties and the building of community connections, both within and across disciplines.
Building on these core elements, there are various implementations of the hackathon concept with respect to the overall purpose, mode of participation, style of work environment and motivation \citep{Drouhard2017}.
''Catalytic`` hackathons seek novel project ideas aimed at solving a tractable, well-defined challenge.
''Contributive`` hackathons seek to improve to an existing effort through focused work on discrete tasks, for example to make up for deficiencies in an ongoing project.
Finally, ''Communal`` hackathons place a strong focus on building a culture of practice and developing resources within an existing community, often defined by a specific domain of knowledge.

Our past hack week events most closely follow the communal hackathon model as it applies to scientific communities of practice.
Our approach aims to combine structured, tutorial-style instruction with informal education and peer learning opportunities occurring within projects and hacks.
Within the communal model we see these tools being implemented across a spectrum of approaches, the design of which depends on the specific characteristics of each community of practice (Figure 1).
For example, the astronomy community is relatively small and has a foundation of shared approaches and software implementations, allowing for a greater focus on project work over formal tutorials.
In contrast, both the neuro- and geoscience communities covered a broader range of sub-disciplines and had a less cohesive set of existing practices, calling for greater focus on tutorials and education.

We note that the terminology for these events is constantly evolving, and that the ''hackathon`` concept may have implicit connotations that are disfavored in some communities.
One criticism of hackathons is that they propel the ''geek`` stereotype and may present a barrier to creating an inclusive working environment, especially for women \citep{Decker2015}.
Another problem is the competitive atmosphere of many industry hackathons, where teams actively compete for prizes.
Because certain groups are more risk-averse than others, and risk-aversion traces demographics like gender, we urge caution when contemplating making an event competitive.
Other related nomenclature includes the ''unconference`` which is also a time-bounded event gathering together a specific community, but these often have minimal structure and do not necessarily have a focus on software.
The ''sprint`` or ''scrum`` label typically refers to an event focused on rapid software development on a specific set of code.
%This may be a component of our hack week model but it ignores the important pedagogical components.
