
\section*{Conclusions}

The fast-paced changes of the computational and methodological landscape require traditional fields of science to rapidly adapt to new data analysis challenges.
Traditional modes of learning, including university curricula, are often too slow to incorporate new developments on short enough time scales to meet their acute need in scientific advancement.
To address this imbalance, new types of workshops, including unconferences, hackathons and bootcamps, have been developed in recent years in various scientific disciplines to exist alongside with and support the existing structure of academic conferences.
Here, we introduce one such concept, hack weeks, and detail the underlying philosophical ideas along with experiences from events held in three different fields

As introduced above, hack weeks serve multiple purposes, including dissemination of state-of-the-art technological advances through the scientific community, building collaborations between academic subdisciplines and fostering interdisciplinary research as well as  promoting open science and reproducibility.
Initial results from three events held in 2016 \textbf{and 2017} in three different fields (astronomy, geosciences and neurosciences) indicate that hack weeks succeed at all of these objectives, but that the measure of success is field-specific in that it depends to some degree on how much the concepts hack weeks promote were already adopted within the community.
Hack weeks are still a very young concept, and estimating the long-term impact of these events within the scientific communities they serve will require follow-up over multiple years to asses their effect on collaboration networks, career outcomes for early-career academics and adoption of new methods.
We have shown, however, that hack weeks provide an easy-to-implement, fairly low-cost method to introduce new technologies and methods into scientific fields on much shorter time scales than traditional teaching efforts can.
