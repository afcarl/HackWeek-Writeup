\section*{Themes}

To date, all organized hack weeks have been subject-specific, i.e.\ aimed to bring together a community with a shared scientific interest, such as neuroscience.
Advantages to this approach include shared language and scientific objectives within communities organized by subject, leaving more time for active collaboration on cutting-edge science.
On the other hand, homogeneity may lead to \textit{group think} and inhibit new, creative solutions. 
In this case, it may be advantageous to design a hack week around a technique (e.g.\ Gaussian Processes) or modality (e.g.\ imaging), such as the ImageXD (image processing across domains\footnote{\url{http://http://www.imagexd.org/}} meetings. 
For these events, building a shared vocabulary and understanding of major data analysis problems is crucial, but they also allow for cross-disciplinary diffusion of techniques into other subjects and therefore decrease the risk of duplication of method development efforts.

%This has several clear advantages.
%In particular, communities organized by subject generally share the same language.%, which drastically cuts down on the time participants spend simply agreeing on scientific terms.
%Similarly, the scientific objectives and questions are broadly clear to all participants.%: there is little need to explain in detail why a particular question may be of interest to the field.
%Finally, there are practical considerations: while even within fields the use of particular data sets and software packages may differ quite widely, there is generally a shared understanding of the type of data usually taken, and the most common ways to analyze this data.
%This leaves more time for active collaboration on cutting-edge science.

%However, there are also drawbacks to limiting the theme of a hack week to a single field.
%Homogeneity may be a disadvantage if it leads to \textit{group think} and inhibits new, creative solutions.
%Within a field, the chance of participants being familiar with each other may be larger, leading to the formation of cliques and insufficient mixing.
%These disadvantages may be mitigated by organizing a hack week around a technique (e.g. Gaussian Processes) or modality (e.g. imaging) instead.
%In these case, building a shared vocabulary and understanding the major data analysis problems in each field is a crucial task.%, and should be allocated sufficient time in form of introductory talks or tutorials to foster cross-disciplinary collaboration.
%On the other hand, cross-disciplinary hack weeks around a specific technique or type of data set allow methods developed in a specific field to diffuse into other subjects and therefore help avoid duplication of method development efforts.

%An example of such events are the ImageXD (image processing across domains\footnote{\url{http://http://www.imagexd.org/}} meetings held at the UC Berkeley Institute for Data Science in the Summer 2016, and at the UW eScience Institute in the Spring of %2017.
%These three-day events also included a mix of tutorials, talks, and joint work on projects, but in contrast to the hack weeks described in the present paper, these events included participants from a variety of different research fields, with the common thread being that they all have an interest in the analysis of images, as part their research.
%Thus, audience and speakers included researchers from wide array of domains: astronomy, neuroscience and geosciences, but also material science, chemistry, biology, medicine, engineering, and others.
%In addition, participants included computer vision researchers, who develop algorithms that can be applied to answer scientific questions.
%Similarly, a series of events at UC Berkeley (Text XD) focused on the analysis of text data across different domains of research.
