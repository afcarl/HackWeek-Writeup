
\section*{Conclusions}

The fast-paced changes in the computational and methodological landscape require that traditional fields of science rapidly adapt to new data analysis challenges.
To address these challenges, new types of workshops, including unconferences, hackathons and bootcamps have been developed in recent years in various scientific disciplines and now exist alongside with, and support the existing structure of academic conferences, formal classes, and other learning opportunities.
Here, we introduce one such concept, hack weeks, and detail the underlying philosophical ideas along with experiences from events held in three different fields.

Hack weeks serve multiple purposes, including dissemination of technological advances through the scientific community, building collaborations between academic subdisciplines and fostering interdisciplinary research. Initial results from three events held in 2016 and 2017 in three different fields (astronomy, geosciences and neurosciences) indicate that hack weeks succeed at all of these objectives.

Hack weeks are still a very young concept, and estimating the long-term impact of these events within the scientific communities they serve will require follow-up over multiple years, to assess their effect on collaboration networks, career outcomes and adoption of new methods.
We have shown, however, that hack weeks provide an easy-to-implement, fairly low-cost method to introduce new technologies and methods into scientific fields on much shorter time scales than traditional teaching efforts.
While we focus here on hack weeks in scientific fields, the concept could be extended to other areas, and is more generally useful in any area (1) where useful tools can be learned in short tutorials, (2) where results and outcomes can be produced on the timescale of a few days, and (3) that would benefit from collaborative approaches that cross traditional boundaries. Such areas could include the social sciences, the humanities, as well as music and art.