\section*{Audience and Participant Selection}

Hack weeks differ from traditional conferences or summer schools in that knowledge transfer occurs across many levels of seniority, disciplinary boundaries, and novelty of the topics discussed.
In addition, a substantial amount of hack week content is generated during the event itself, requiring active participation from attendees.
Therefore in order to maximize learning outcomes and collaborative exchanges, it is crucial that the participant selection process be carried out with considerable care.

In our experience, a participant group that is diverse across categories of minority status, geographical origin, gender, discipline and career track helps to ensure we meet these objectives.
To achieve this diversity, we advocate for a selection process that is as quantitative and transparent as possible, enabling participants to hold organizers accountable for their selection decisions.
Research particularly in the hiring literature has shown that cohort selection is most effective and unbiased when selection procedures are as quantitative as possible~\cite{sunstein2015wiser}. 

Participant selection is an intrinsically difficult process that is fraught with personal and structural biases~\cite[e.g.][]{sunstein2015wiser}. It requires laying out a definition of successful participation, what criteria must be met in order to maximize the likelihood of success, and how those criteria will be assessed given the information about the candidates selected during the application stage.  
Traditional selection processes rely heavily on internal heuristics of reviewers, including using unrelated characteristics as a cipher for suitability. On the other hand, while purely mechanistic procedures may to be able to mitigate some reviewer biases, this may come at a cost of baking in structural inequalities. 

For hack weeks, it is not obvious a priori that prerequisites for attendance in terms of skills should exist, and prerequisites will depend on the objectives of the workshop. For example, Astro Hack Week has traditionally accepted participants at all skill levels with respect to data science and did not a merit-based selection, whereas Neuro Hack Week did include skill-based criteria in their selection (see also SM, Section 2.1 for more detail on the individual selection procedures). If merit-based selection is part of the process, organizers face the crucial decision whether to assess merit blindly or not. Research particularly around hiring procedures has shown that there is no single correct procedure. 

Because human decision makers tend to be swayed by unrelated characteristics including name~\cite{bertrand2004} or gender~\cite{mossracusin2012}, an initial merit selection blinded to demographic characteristics can be an effective way to counteract certain biases related to internal heuristics. A merit selection could then be performed via scores given independently by members of the organizing committee based on a set of pre-defined, explicit selection criteria. This type of blinded procedure tends to reduce biases during participant selection when committees would otherwise not consider diversity during their selection [REF]. However, a blind selection purely on a figure of merit will likely still be biased if some groups of applicants have been systematically disadvantaged in certain ways: for example, if workshop organizers require a minimum level of programming experience, this could disadvantage candidates who have had fewer opportunities to learn programming due to structural inequalities. 

Blinding to demographics can be counterproductive if it excludes participants who might have had less exposure to certain technologies or fewer opportunities to learn certain skills. Additionally, blinding has been found to have negative effects on diversity for committees that have already had a strong commitment to diversity, because they often correct for structural inequalities by considering demographic variables during merit selection~\cite{behaghel2015unintended}.
In this case, it may be beneficial to construct selection criteria that explicitly consider diversity and inclusivity (as Neuro Hack Week has done; see also SM, Section 2.1). 
Because systemic biases likely also enter at the application stage (where underrepresented groups may be less likely to apply) organizers should consider oversampling traditionally disenfranchised groups compared to the population of applicants. 

No matter the selection procedures used, we encourage organizers to critically examine their cohort selection, experiment with new approaches, and routinely evaluate their procedure. For example, comparing demographic characteristics of the selected versus non-selected groups can unveil unintended biases during the merit-selection base and thus allow adjustments in the procedure to mitigate or fully remove these effects.
