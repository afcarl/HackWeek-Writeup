\begin{abstract}
Across almost all scientific disciplines, instruments that record experimental data and methods required for storage and data analysis are rapidly increasing in complexity.
As a result of these rapid developments, scientific communities need to adapt on shorter time scales than traditional university curricula can evolve. New modes of knowledge transfer are therefore needed.
On the other hand, the broad applicability of data science tools to a range of problems has generated new opportunities to foster exchange of ideas and computational workflows across disciplines.
In recent years, hack weeks have emerged as a novel and effective tool for fostering these exchanges and providing training in modern data analysis workflows.
While there are variations in hack week implementation, all events consist of a common core of three components: tutorials in state-of-the-art methodology, peer-learning and project work in a collaborative environment.
In this paper, we present the concept of a hack week in the larger context of scientific meetings.
We motivate the need for such an event and present its strengths and challenges.
We find that hack weeks are successful at cultivating collaboration and facilitating the exchange of knowledge.
Participants self-report that these events help them both in their day-to-day research as well as their careers.
Based on our results, we conclude that hack weeks present an effective, easy-to-implement, fairly low-cost tool to positively impact data analysis literacy in academic disciplines, foster collaboration and cultivate best practices.
\end{abstract}
