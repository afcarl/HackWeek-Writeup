\section*{What is a hack week?}

Our hack weeks combine structured, tutorial-style instruction with informal education and peer learning opportunities occurring within projects and hacks. 
Our focus on both pedagogy and open project work aims to fill an educational space we feel is not currently addressed by existing models (Figure \ref{fig:hackspectrum}). 
The hackathon, a time-bounded, collaborative events that bring together participants around a shared challenge or learning objective \cite{Decker2015}, forms one primary axis of our events. 
Hackathons originated from the open-source software movement and have historically focused on software and technology development. 
In recent years hackathons have evolved into a model providing opportunities for intensive, interdisciplinary collaboration \cite{Groen2015-cj} and education \cite{Kienzler2015-zu,Lamers2014-xf} in the sciences. 
Core element of all hackathons include opportunities for networking, strengthening of social ties, and the building of community connections, both within and across disciplines.
Building on these core elements hackathons have been implemented in different ways depending on the overall purpose, mode of participation, style of work environment and participant motivation \cite{Drouhard2017}. 
``Catalytic'' hackathons seek novel project ideas aimed at solving a tractable, well-defined challenge.
``Contributive'' hackathons seek to improve to an existing effort through focused work on discrete tasks, for example to make up for deficiencies in an ongoing project.
Finally, ``Communal'' hackathons place a strong focus on building a culture of practice and developing resources within an existing community, often defined by a specific domain of knowledge.

Our hack weeks extend the scientifically-focused, communal hackathon model into a space that includes a strong element of pedagogy and peer learning, forming the second axis of our events. 
Teaching within our events draws from a spectrum of approaches such as those traditionally employed in summer schools and software workshops. 
However, by embedding these approaches in a more open hackathon framework, we can explore unique ways of enhance the traditional summer school model.
For example, our hack weeks transfer knowledge across many levels of seniority, with participants routinely leading tutorials, and with numerous opportunities to `learn by doing' within a constructivist educational framework \cite{Bransford2000-lu}.
Additionally, the content of our tutorials tends to favor cutting-edge topics and experimental approaches over established knowledge. 
Finally, the social structure of hackathons enable us to prioritize participant diversity thereby maximizing the likelihood for the co-creation of learning during the event. 

We have found that the multiple dimensions of the hack week model provide flexibility across a spectrum of approaches, the design of which depends on the specific characteristics of each community of practice.
For example, the astronomy community is relatively small and has a foundation of shared approaches and software implementations, allowing for a greater focus on project work over formal tutorials.
In contrast, both the neuro- and geoscience communities covered a broader range of sub-disciplines and had a less cohesive set of existing practices, calling for greater focus on tutorials and education.

We note that the terminology for these events is constantly evolving, and that the ``hackathon'' concept may have implicit connotations that are disfavored in some communities.
