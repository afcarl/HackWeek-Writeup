\section*{Design considerations}

Scheduling, group size and venue are important design considerations contributing to the success of a hack week.   
Longer events allow for a larger taught component, more ambitious projects and cross-disciplinary exchanges. 
By spending more time together, participants are more likely to overcome barriers of professional terminology.
However, events that are too long may lead to fatigue among attendees, resulting in a drop in positive outcomes later in the workshop.
A well-designed hack week will have a clear schedule that limits the number of parallel sessions, in order to avoid decision fatigue, and will balance the duration of taught components and project work. 

A hackweek requires a flexible workspace environment with ample opportunity for re-configuration. 
Participants must have access to rooms that can accommodate lectures combined with interactive exchanges and individual work on laptops. 
Workspaces must also accommodate interactive project work where small teams can gather and work together.    
%Universities are an obvious first choice for hosting a hack week given their access to scientific resources, research support and infrastructure. 
%However traditional university lecture halls often do not meet the hack week criteria for interactive exchanges.
Fortunately, many universities are experimenting with other types of spaces: for example, the adoption of active learning teaching methods \cite{prince2004} has led to the development of modular classrooms, designed for group activities and flexible seating arrangements.

Another important design consideration is group size.
If the group is too large, chances for random participant exchanges are reduced, and knowledge transfer may decline as the workshop fractures into smaller groups, often among participants who already know each other.
If the group is too small, reaching the desired level of diversity among participants to foster new collaborations across sub-fields and disciplines may require much more careful managing of the participant selection than is true for larger groups.
Our events have consisted of groups of around 50-70 people, which we have found anecdotally to be large enough to encourage a breadth of projects while allowing the workshop to function as a cohesive group.

%As mentioned above, the balance between pedagogy and open project work depends both on the goals of the workshop and the topics around which the workshop is organized.
%If participants have little shared knowledge, more teaching may be necessary in order to allow participants to effectively communicate with each other.
%In communities where a shared understanding exists, tutorials can be shortened to focus on more advanced or innovative topics, leaving more time for active participation.

This breadth of possible workshop outcomes makes it difficult to design for all possible participant goals, and calls for adaptive, flexible leadership among hack week organizers.
The large variety in participant backgrounds and experiences--and the resulting range in personalities of attendees--requires careful, active facilitation of both taught and project components of a hack week. Community building is a core component of a hack week, and designs need to include both very strong personalities and very shy participants. In particular the presence of impostor syndrome experienced by many participants must be taken into account during workshop design (see also supplementary materials for concrete suggestions).
%can lead to an increase in the prevalence of impostor syndrome experienced by many participants (see also supplementary materials), and it 
